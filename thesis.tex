\documentclass[]{mythesis}

%% -------------------------------------
%% Standard packages
%% -------------------------------------

%\usepackage{layout} 

%% -------------------------------------
%% Config
%% -------------------------------------
%\linespread{1.3}

%% PDF metadata
\makeatletter
\@ifpackageloaded{hyperref}{%
  \hypersetup{%
    pdftitle = {A search for supersymmetry using the \aT variable with the CMS
        detector and the impact of experimental searches for supersymmetry on
    supersymmetric parameter space},
    pdfsubject = {Particle physics},
    pdfkeywords = {physics, LHC, CMS, SUSY, supersymmetry},
    pdfauthor = {Samuel Rogerson}
  }
}{}
\makeatother

% Path to look for graphics in
\graphicspath{{figures/}{gen_figures/}}

%% Define the thesis title and author
\title{A search for supersymmetry using the \aT variable with the CMS detector
    and the impact of experimental searches for supersymmetry on supersymmetric
parameter space}
%%\title{t}
\author{Samuel Rogerson}
%% -------------------------------------
%% Content
%% -------------------------------------

%% Start the document
\begin{document}
%\layout

%% Define the un-numbered front matter (cover pages, rubrik and TOC)
%\frontmatter
  %\input{chapters/front.tex}

%% Start the content body of the thesis
%\mainmatter
\setcounter{page}{1}
\pagenumbering{roman}
  \maketitle
  \begin{abstract}

      A search for supersymmetry in final states with jets and missing
      transverse energy is performed in pp collisions at a centre-of-mass energy
      of $\sqrt{s}=7\TeV$. The data sample corresponds to $4.98\invfemtobarn$
      and $11.7\invfemtobarn$, for $\sqrt{s}=7\TeV$ and $\sqrt{s}=8\TeV$
      respectively, collected by the CMS experiment at the LHC.  A dimensionless
      kinematic variable is used as the main discriminant between genuine and
      misreconstructed signal events. The search is performed in a signal region
      binned according to the scalar sum of the transverse energy of jets and
      the number of jets identified as originating from a bottom quark.  The
      limits are presented in the parameter space of the \ac{cMSSM} as well as
      in simplified models with particular attention paid to compressed spectra
      and third-generation models.

      Global frequentist fits to the \ac{cMSSM} and NUHM1 are also performed
      using the \Mastercode framework incorporating recent experimental
      constraints, including those presented here.  Global likelihood contours
      are presented in the parameter planes of both the \ac{cMSSM} and NUHM1, as
      well as a selection of 1D likelihood functions for observables.
  
  \end{abstract}
  
  \tableofcontents \listoffigures

\newpage
\pagenumbering{arabic}
\setcounter{page}{1}
  \include{chapters/introduction}
  \include{chapters/theory}
  \include{chapters/detector}
  \include{chapters/analysis}
  \include{chapters/constraining_models}
  \include{chapters/future_prospects}
  \include{chapters/conclusion}

%% Produce the appendices
%\appendix
  %\include{chapters/appendix}
%
%% Produce the un-numbered back matter (e.g. colophon,
%% bibliography, tables of figures etc., index...)
%\backmatter
  %\input{chapters/end}
  \bibliographystyle{plain}
  \bibliography{thesis}

\end{document}
