\documentclass[]{mythesis}
\usepackage[utf8]{inputenc}
\DeclareUnicodeCharacter{00A0}{ }
\usepackage[style=numeric,bibstyle=numeric,mcite=true]{biblatex}
\addbibresource{thesis.bib}

\defbibentryset{Svenetal}{Svenetal1,Svenetal2}
\defbibentryset{g-2}{g-21,g-22,g-23,g-24,g-25,g-26,g-27}
\defbibentryset{FeynHiggs}{FeynHiggs1,FeynHiggs2,FeynHiggs3,FeynHiggs4}
\defbibentryset{newBNL}{newBNL1,newBNL2}
\defbibentryset{bsgth}{bsgth1,bsgth2,bsgth3,bsgth4,bsgth5}
\defbibentryset{SuFla}{SuFla1,SuFla2}

\linespread{1.3}

%% PDF metadata
\makeatletter
\@ifpackageloaded{hyperref}{%
  \hypersetup{%
    pdftitle = {A search for supersymmetry using the \aT variable with the CMS
        detector and the impact of experimental searches for supersymmetry on
    supersymmetric parameter space},
    pdfsubject = {Particle physics},
    pdfkeywords = {physics, LHC, CMS, SUSY, supersymmetry},
    pdfauthor = {Samuel Rogerson}
  }
}{}
\makeatother

% Path to look for graphics in
\graphicspath{{figures/}{gen_figures/}}

%% Define the thesis title and author
\title{A search for supersymmetry using the \aT variable with the CMS detector
    and the impact of experimental searches for supersymmetry on supersymmetric
parameter space}
\author{Samuel Rogerson}

\begin{document}

\setcounter{page}{1}
    \maketitle
    \begin{abstract}

        A search for supersymmetry in final states with jets and missing
        transverse energy is performed in pp collisions at a centre-of-mass
        energy of $\sqrt{s}=7\TeV$. The data sample corresponds to
        $4.98\invfemtobarn$ collected by the CMS experiment at the LHC in 2011.
        A dimensionless kinematic variable is used as the main discriminant
        between genuine and misreconstructed signal events. The search is
        performed in a signal region binned according to the scalar sum of the
        transverse energy of jets and the number of jets identified as
        originating from a bottom quark.  The limits are presented in the
        parameter space of the \ac{cMSSM} as well as in simplified models with
        particular attention paid to compressed spectra and third-generation
        models.

        Global frequentist fits to the \ac{cMSSM} and NUHM1 are also performed
        using the \Mastercode framework incorporating recent experimental
        constraints, including those presented here.  Global likelihood contours
        are presented in the parameter planes of both the \ac{cMSSM} and NUHM1,
        as well as a selection of 1D likelihood functions for observables.
  
    \end{abstract}
  
    \tableofcontents \listoffigures

    \newpage

    \include{chapters/introduction}
    \include{chapters/theory}
    \include{chapters/detector}
    \include{chapters/analysis}
    \include{chapters/constraining_models}
    \include{chapters/conclusion}

    \printbibliography

\end{document}
